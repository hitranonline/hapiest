\documentclass{article}

\usepackage[utf8]{inputenc}

\usepackage{hyperref}
\hypersetup{
    colorlinks=true,
    linkcolor=blue,
    filecolor=magenta,      
    urlcolor=cyan,
}

\begin{document}

%%\comment{POSTER}

\begin{center}
\section{HITRAN Application Programming Interface and Efficient Spectroscopic Tools (HAPIEST)}
\textbf{
    \underline{J. Karns,$^a$} \underline{W. Matt,$^a$} \underline{R. V. Kochanov,$^b$}  \underline{I.~E. Gordon,$^b$}
    \underline{L.~S.~Rothman}, \underline{$^b$, Y.~Tan, $^b$}, R.~Hashemi,$^{b,c}$ S.~Kanbur, $^a$  B. Tenbergen,$^a$
}
\end{center}

\begin{center}
$^a$\textit{State University of New York at Oswego, Oswego, NY, USA}
$^b$\textit{Atomic and Molecular Physics, Harvard-Smithsonian Center for Astrophysics, Cambridge, MA, USA}
$^c$\textit{ University of Lethbridge, Alberta, Canada}
\end{center}

Current high-resolution spectroscopic data have become more and more complex and extensive. In order to make a
connection between the spectroscopic data in different formats and a spectra observed in different applications, one
needs a reliable and flexible tool, which is easy to use and deploy.  

The HITRAN Application Programing Interface and Efficient Spectroscopic Tools (HAPIEST) is the joint project which
started in the fall of 2017 as a collaboration between the HITRAN team and the State University of New York (Oswego).
The purpose of HAPIEST is to simplify experiences using the HITRAN Application Programing Interface (HAPI,
\href{http://hitran.org/hapi}{http://hitran.org/hapi}) \footnote{Kochanov RV, Gordon IE, Rothman LS, Wcislo P, Hill C,
Wilzewski JS. HITRAN Application Programming Interface (HAPI): A comprehensive approach to working with spectroscopic
data. J Quant Spectrosc Radiat Transf 2016;177:15–30. doi:10.1016/j.jqsrt.2016.03.005.} to work efficiently with the
HITRAN database HITRAN\footnote{Gordon IE, Rothman LS, Hill C, Kochanov RV, Tan Y, Bernath PF, et al. J. Quant.
Spectrosc. Radiat. Transf. 203, 3-69 (2017).}.  HAPIEST provides cross-platform graphical interactive tools giving
access to the basic features of HAPI such as data fetching and selection, as well as generating and plotting of the
spectral functions (absorption coefficients, transmittance, absorption, and radiance spectra). 

HAPIEST provides interactive access to most of the controls which are involved in the spectral filtering and
simulation, and is distributed both as binary and source code. The recent version of the source code can be found on
Github (\href{https://github.com/hapiest-team/hapiest}{https://github.com/hapiest-team/hapiest}). The HAPI library, on
which the HAPIEST is based, is a free open-source Python module (library) which provides a set of tools for working
with the structured spectroscopic data from different sources. The principal aim of HAPI is facilitating
physically-sound interpretation of observations and more realistic models for a wide variety of applications such as
astrophysics, planetary science, climate simulations, remote sensing, theoretical spectroscopy, and data mining. Having
such a tool is important in particular to prevent possible errors in radiative transfer calculations caused by misuse
of spectroscopic tools and databases. The description of the first version of HAPI and its features can be found in the
dedicated paper\footnote{Kochanov RV, Gordon IE, Rothman LS, Wcislo P, Hill C, Wilzewski JS. J. Quant. Spectrosc.
Radiat. Transf. 177, 15-30 (2016).} and in the official web page (\href{http://hitran.org/hapi}{http://hitran.org/hapi}).

% If you want to include a figure add next three lines and supply figure in .eps format

%\begin{center}
%\includegraphics[height=5.5cm]{Fig.eps}
%\end{center}

% The next line is optional but much appreciated for indexing names (format is {Last name then initials}
\index{author}{Karns J.} \index{author}{Matt W. N.} \index{author}{Kochanov R. V.} \index{author}{Gordon I. E.}
\index{author}{Tan Y.} \index{author}{Hashemi R.}\index{author}{Rothman L. S.} \index{author}{Tenbergen B.}
\index{author}{Kanbur S.}

\end{document}
